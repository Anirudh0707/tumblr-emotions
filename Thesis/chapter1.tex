\chapter{Introduction}

Sentiment analysis has been an active area of research in the past few years, especially on the readily available Twitter data, e.g. Bollen et al. \cite{bollen} who investigated the impact of collective mood states on stock market or Flaxman et al. \cite{seth-twitter} who analysed day-of-week population well-being.

Contrary to Twitter, Tumblr's posts are not limited to 140 characters, allowing more expressiveness, and are not centered on the textual content but on the image content instead. A Tumblr post will almost always be an image with some text accompanying the latter. Pictures have become prevalent on social media and characterising them could enable the understanding of billions of users. 

http://www.ifp.illinois.edu/~jyang29/papers/AAAI15-sentiment.pdf

We propose a novel method to uncover the emotional of an individual posting on social media. The ground truth emotion will be extracted from the tags, considered as the `self-reported' emotion of the user. Our model incorporates both text and image and we aim to `read' them to be able to understand the emotional content they imply about the user.










