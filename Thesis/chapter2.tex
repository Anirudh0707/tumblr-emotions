\chapter{Tumblr data}

%%%%%%%%%%%%%%%%%%%%%%%%%%%%%%%%%%%%%%%%%%%%%%%%%%%%%%%%%%%%
%%%%%%%%%%%%%%%%%%%%  NEW SECTION   %%%%%%%%%%%%%%%%%%%%%%%%
%%%%%%%%%%%%%%%%%%%%%%%%%%%%%%%%%%%%%%%%%%%%%%%%%%%%%%%%%%%%
\section{Overview of the data}
Tumblr's posts were extracted using the official API thanks to their tags that were taken as the ground truth. The tags represent the user's emotion: happy, sad, angry, surprised, scared or disgusted. The data extraction took several weeks due to the API's limitations: 1,000 requests per hour and 5,000 requests per day, with each request containing 20 posts. The final dataset has about one million posts and six different emotions.

Here is an example of a post with its associated image:
